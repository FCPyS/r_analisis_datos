% Options for packages loaded elsewhere
\PassOptionsToPackage{unicode}{hyperref}
\PassOptionsToPackage{hyphens}{url}
\PassOptionsToPackage{dvipsnames,svgnames,x11names}{xcolor}
%
\documentclass[
  letterpaper,
  DIV=11,
  numbers=noendperiod]{scrreprt}

\usepackage{amsmath,amssymb}
\usepackage{iftex}
\ifPDFTeX
  \usepackage[T1]{fontenc}
  \usepackage[utf8]{inputenc}
  \usepackage{textcomp} % provide euro and other symbols
\else % if luatex or xetex
  \usepackage{unicode-math}
  \defaultfontfeatures{Scale=MatchLowercase}
  \defaultfontfeatures[\rmfamily]{Ligatures=TeX,Scale=1}
\fi
\usepackage{lmodern}
\ifPDFTeX\else  
    % xetex/luatex font selection
\fi
% Use upquote if available, for straight quotes in verbatim environments
\IfFileExists{upquote.sty}{\usepackage{upquote}}{}
\IfFileExists{microtype.sty}{% use microtype if available
  \usepackage[]{microtype}
  \UseMicrotypeSet[protrusion]{basicmath} % disable protrusion for tt fonts
}{}
\makeatletter
\@ifundefined{KOMAClassName}{% if non-KOMA class
  \IfFileExists{parskip.sty}{%
    \usepackage{parskip}
  }{% else
    \setlength{\parindent}{0pt}
    \setlength{\parskip}{6pt plus 2pt minus 1pt}}
}{% if KOMA class
  \KOMAoptions{parskip=half}}
\makeatother
\usepackage{xcolor}
\setlength{\emergencystretch}{3em} % prevent overfull lines
\setcounter{secnumdepth}{5}
% Make \paragraph and \subparagraph free-standing
\ifx\paragraph\undefined\else
  \let\oldparagraph\paragraph
  \renewcommand{\paragraph}[1]{\oldparagraph{#1}\mbox{}}
\fi
\ifx\subparagraph\undefined\else
  \let\oldsubparagraph\subparagraph
  \renewcommand{\subparagraph}[1]{\oldsubparagraph{#1}\mbox{}}
\fi


\providecommand{\tightlist}{%
  \setlength{\itemsep}{0pt}\setlength{\parskip}{0pt}}\usepackage{longtable,booktabs,array}
\usepackage{calc} % for calculating minipage widths
% Correct order of tables after \paragraph or \subparagraph
\usepackage{etoolbox}
\makeatletter
\patchcmd\longtable{\par}{\if@noskipsec\mbox{}\fi\par}{}{}
\makeatother
% Allow footnotes in longtable head/foot
\IfFileExists{footnotehyper.sty}{\usepackage{footnotehyper}}{\usepackage{footnote}}
\makesavenoteenv{longtable}
\usepackage{graphicx}
\makeatletter
\def\maxwidth{\ifdim\Gin@nat@width>\linewidth\linewidth\else\Gin@nat@width\fi}
\def\maxheight{\ifdim\Gin@nat@height>\textheight\textheight\else\Gin@nat@height\fi}
\makeatother
% Scale images if necessary, so that they will not overflow the page
% margins by default, and it is still possible to overwrite the defaults
% using explicit options in \includegraphics[width, height, ...]{}
\setkeys{Gin}{width=\maxwidth,height=\maxheight,keepaspectratio}
% Set default figure placement to htbp
\makeatletter
\def\fps@figure{htbp}
\makeatother

\KOMAoption{captions}{tableheading}
\makeatletter
\makeatother
\makeatletter
\@ifpackageloaded{bookmark}{}{\usepackage{bookmark}}
\makeatother
\makeatletter
\@ifpackageloaded{caption}{}{\usepackage{caption}}
\AtBeginDocument{%
\ifdefined\contentsname
  \renewcommand*\contentsname{Table of contents}
\else
  \newcommand\contentsname{Table of contents}
\fi
\ifdefined\listfigurename
  \renewcommand*\listfigurename{List of Figures}
\else
  \newcommand\listfigurename{List of Figures}
\fi
\ifdefined\listtablename
  \renewcommand*\listtablename{List of Tables}
\else
  \newcommand\listtablename{List of Tables}
\fi
\ifdefined\figurename
  \renewcommand*\figurename{Figure}
\else
  \newcommand\figurename{Figure}
\fi
\ifdefined\tablename
  \renewcommand*\tablename{Table}
\else
  \newcommand\tablename{Table}
\fi
}
\@ifpackageloaded{float}{}{\usepackage{float}}
\floatstyle{ruled}
\@ifundefined{c@chapter}{\newfloat{codelisting}{h}{lop}}{\newfloat{codelisting}{h}{lop}[chapter]}
\floatname{codelisting}{Listing}
\newcommand*\listoflistings{\listof{codelisting}{List of Listings}}
\makeatother
\makeatletter
\@ifpackageloaded{caption}{}{\usepackage{caption}}
\@ifpackageloaded{subcaption}{}{\usepackage{subcaption}}
\makeatother
\makeatletter
\@ifpackageloaded{tcolorbox}{}{\usepackage[skins,breakable]{tcolorbox}}
\makeatother
\makeatletter
\@ifundefined{shadecolor}{\definecolor{shadecolor}{rgb}{.97, .97, .97}}
\makeatother
\makeatletter
\makeatother
\makeatletter
\makeatother
\ifLuaTeX
  \usepackage{selnolig}  % disable illegal ligatures
\fi
\IfFileExists{bookmark.sty}{\usepackage{bookmark}}{\usepackage{hyperref}}
\IfFileExists{xurl.sty}{\usepackage{xurl}}{} % add URL line breaks if available
\urlstyle{same} % disable monospaced font for URLs
\hypersetup{
  pdftitle={R para el análisis estadístico de datos},
  pdfauthor={Ana Escoto},
  colorlinks=true,
  linkcolor={blue},
  filecolor={Maroon},
  citecolor={Blue},
  urlcolor={Blue},
  pdfcreator={LaTeX via pandoc}}

\title{R para el análisis estadístico de datos}
\author{Ana Escoto}
\date{2024-10-06}

\begin{document}
\maketitle
\ifdefined\Shaded\renewenvironment{Shaded}{\begin{tcolorbox}[frame hidden, breakable, sharp corners, enhanced, interior hidden, borderline west={3pt}{0pt}{shadecolor}, boxrule=0pt]}{\end{tcolorbox}}\fi

\renewcommand*\contentsname{Table of contents}
{
\hypersetup{linkcolor=}
\setcounter{tocdepth}{2}
\tableofcontents
}
\bookmarksetup{startatroot}

\hypertarget{sobre-el-curso}{%
\chapter*{Sobre el curso}\label{sobre-el-curso}}
\addcontentsline{toc}{chapter}{Sobre el curso}

\markboth{Sobre el curso}{Sobre el curso}

\hypertarget{introducciuxf3n-a-r-y-rstudio-4-horas}{%
\section*{Introducción a R y Rstudio (4
horas)}\label{introducciuxf3n-a-r-y-rstudio-4-horas}}
\addcontentsline{toc}{section}{Introducción a R y Rstudio (4 horas)}

\markright{Introducción a R y Rstudio (4 horas)}

\emph{Objetivo: que el estudiantado sea familiarice con la interfase de
trabajo y la programación por objetos, del mismo modo que sean capaces
de realizar tareas básicas tales como crear un script, un proyecto,
objetos, ambientes e instalar paqueterías}.

\hypertarget{importaciuxf3n-de-informaciuxf3n-y-primera-revisiuxf3n-de-fuentes-demogruxe1ficas-4-horas}{%
\section*{Importación de información y primera revisión de fuentes
demográficas (4
horas)}\label{importaciuxf3n-de-informaciuxf3n-y-primera-revisiuxf3n-de-fuentes-demogruxe1ficas-4-horas}}
\addcontentsline{toc}{section}{Importación de información y primera
revisión de fuentes demográficas (4 horas)}

\markright{Importación de información y primera revisión de fuentes
demográficas (4 horas)}

\begin{enumerate}
\def\labelenumi{\alph{enumi}.}
\item
  Importación de información a R en diferentes formatos
\item
  Revisión de encuestas y ejemplos de importación de datos en formatos
  diferentes
\item
  Revisión preliminar y limpieza de información
\end{enumerate}

\emph{Objetivo: que el estudiantado sea capaz de importar información
desde diferentes formatos (.txt, .csv, .xlsx, .dta, .dbf) a R, así como
de exportar sus resultados en estos formatos. Del mismo modo que sean
capaces de revisar de manera preliminar los objetos de tipo
\texttt{data.frame}: funciones \texttt{dplyr::glimpse()},
\texttt{skimr::skim()}, manejo de etiquetas y hacer subconjuntos de
información}

\hypertarget{revisiuxf3n-de-elementos-estaduxedsticos-buxe1sicos-desde-tidyverse-8-horas}{%
\section*{Revisión de elementos estadísticos básicos desde ``tidyverse''
(8
horas)}\label{revisiuxf3n-de-elementos-estaduxedsticos-buxe1sicos-desde-tidyverse-8-horas}}
\addcontentsline{toc}{section}{Revisión de elementos estadísticos
básicos desde ``tidyverse'' (8 horas)}

\markright{Revisión de elementos estadísticos básicos desde
``tidyverse'' (8 horas)}

\begin{enumerate}
\def\labelenumi{\alph{enumi}.}
\item
  Tabulados con \texttt{janitor::tabyl()} y uso de factores de expansión
  con \texttt{pollster::topline()}, \texttt{pollstter::crosstab}.
  Lectura e interpretación de tablas de doble entrada
\item
  Estadística descriptiva básica (medidas de tendencia central,
  dispersión y de posición) con el paquete \texttt{\{dplyr\}}
\item
  Gráficos univariados y bivariados usando \texttt{\{ggplot2\}} y
  extensiones de \texttt{\{ggplot2\}}
\item
  Fusionado de información
\end{enumerate}

\emph{Objetivo: que el estudiantado sea capaz de realizar análisis
estadísticos básicos utilizando encuenstas mexicanas}

\hypertarget{estimaciones-por-intervalo-y-diseuxf1o-complejo-muestral-4-horas}{%
\section*{Estimaciones por intervalo y diseño complejo muestral (4
horas)}\label{estimaciones-por-intervalo-y-diseuxf1o-complejo-muestral-4-horas}}
\addcontentsline{toc}{section}{Estimaciones por intervalo y diseño
complejo muestral (4 horas)}

\markright{Estimaciones por intervalo y diseño complejo muestral (4
horas)}

\begin{enumerate}
\def\labelenumi{\alph{enumi}.}
\item
  Estimaciones para medias
\item
  Estimaciones para proporciones
\item
  Estimaciones para totales
\item
  Estimaciones lineales de coeficientes
\end{enumerate}

\emph{Objetivo: que el estudiantado sea capaz de realizar intervalos de
confianza, calculo de errores estándar con diseño muestral complejo y
sepa evaluar la confiabilidad de sus estimaciones}

\bookmarksetup{startatroot}

\hypertarget{facilitadora}{%
\chapter*{Facilitadora}\label{facilitadora}}
\addcontentsline{toc}{chapter}{Facilitadora}

\markboth{Facilitadora}{Facilitadora}

\hypertarget{ana-ruth-escoto-castillo}{%
\section*{Ana Ruth Escoto Castillo}\label{ana-ruth-escoto-castillo}}
\addcontentsline{toc}{section}{Ana Ruth Escoto Castillo}

\markright{Ana Ruth Escoto Castillo}

Profesora de tiempo completo en la Facultad de Ciencias Políticas y
Sociales, UNAM. Doctora en Estudios de Población por El Colegio de
México, cuenta con el nivel I en el Sistema Nacional de Investigadores.
Coorganizadora del capítulo de la CDMX de la iniciativa global Rladies.
Le interesa el bienestar de la población, en el presente, analizando los
procesos de desigualdad y exclusión en los mercados laborales
latinoamericanos; y, en el futuro, a través del estudio de la
sustentabilidad. Su experiencia en el ámbito académico se ha concentrado
en el estudio de este bienestar, específicamente en el análisis
sociodemográfico de las condiciones laborales y la vinculación del
comercio exterior con el mercado de trabajo, en la relación del cambio
climático y la distribución de ingresos, el consumo energético de los
hogares y sus implicaciones ambientales. Posee experiencia en
recolección de información estadística, diseño y control de procesos de
recolección y su procesamiento. Ha aplicado diversos métodos y
herramientas multivariadas, homologación de información y comparabilidad
de fuentes en sus investigaciones, así como usa de diversos softwares
estadísticos, y ha impartido clases de estadítica aplicada a nivel de
licenciatura y posgrado.

\bookmarksetup{startatroot}

\hypertarget{instalaciuxf3n-de-r-y-rstudio}{%
\chapter*{Instalación de R y
Rstudio}\label{instalaciuxf3n-de-r-y-rstudio}}
\addcontentsline{toc}{chapter}{Instalación de R y Rstudio}

\markboth{Instalación de R y Rstudio}{Instalación de R y Rstudio}

\hypertarget{introducciuxf3n-a-r}{%
\section*{Introducción a R}\label{introducciuxf3n-a-r}}
\addcontentsline{toc}{section}{Introducción a R}

\markright{Introducción a R}

\url{https://youtu.be/YkN5urybh2A}

\hypertarget{instalaciuxf3n-en-os}{%
\section*{Instalación en OS}\label{instalaciuxf3n-en-os}}
\addcontentsline{toc}{section}{Instalación en OS}

\markright{Instalación en OS}

\begin{enumerate}
\def\labelenumi{\arabic{enumi}.}
\tightlist
\item
  Necesito que instalen la versión más nueva de R: Download R-4.4.0 of
  MAC. \emph{The R-project for statistical computing}.
  \url{https://cran.r-project.org/bin/macosx/}
\end{enumerate}

Elije la versión de acuerdo a tu procesador, intel o ARM.

\begin{enumerate}
\def\labelenumi{\arabic{enumi}.}
\setcounter{enumi}{1}
\tightlist
\item
  Instalar también las herramientas Quartz, xcode y fortran
\end{enumerate}

\begin{itemize}
\item
  \url{https://www.xquartz.org/}
\item
  \url{https://developer.apple.com/xcode/resources/}
\item
  \url{https://mac.r-project.org/tools/gfortran-12.2-universal.pkg}
\end{itemize}

\begin{enumerate}
\def\labelenumi{\arabic{enumi}.}
\setcounter{enumi}{2}
\tightlist
\item
  Después de eso instalar el Rstudio, que hoy se encuentra alojado en el
  sitio posit, que vaya acorde con MAC
\end{enumerate}

\url{https://posit.co/download/rstudio-desktop/}

Algunas indicaciones en video, pero son algo viejitas y pueden cambiar
las versiones de R.

\url{https://youtu.be/icWV8jzYOtA}

Algunas indicaciones en video, pero son algo viejitas y pueden cambiar
las versiones de R.

\hypertarget{instalaciuxf3n-en-pc}{%
\section*{Instalación en PC}\label{instalaciuxf3n-en-pc}}
\addcontentsline{toc}{section}{Instalación en PC}

\markright{Instalación en PC}

\begin{enumerate}
\def\labelenumi{\arabic{enumi}.}
\item
  Necesito que instalen la versión más nueva de R: Download R-4.4.0 for
  Windows. \emph{The R-project for statistical computing}.
  \url{https://cran.r-project.org/bin/windows/base/}
\item
  Instalar también la herramienta RTools
  \url{https://cran.r-project.org/bin/windows/Rtools/rtools44/rtools.html}
\item
  Después de eso instalar el Rstudio, que hoy se encuentra alojado en el
  sitio posit, que vaya acorde con Windows
  \url{https://posit.co/download/rstudio-desktop/}
\end{enumerate}

Algunas indicaciones en video, pero son algo viejitas y pueden cambiar
las versiones de R.

\url{https://youtu.be/TNSQikMfgJI}

\hypertarget{ojo}{%
\section*{Ojo}\label{ojo}}
\addcontentsline{toc}{section}{Ojo}

\markright{Ojo}

Desde octubre de 2022, RStudio se volvió \textbf{``Posit''}



\end{document}
